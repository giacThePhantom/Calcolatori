\chapter{Input-output}
I dispositivi di I/O sono necessari al computer per comunicare con l'esterno, devono essere espandibili ed eterogenei. La tipologia di prestazione varia dai casi: 
nel caso di tastiere e mouse interessa pi\`u la latenza, nel caso di dischi o interfacce di rete interessa pi\`u il throughtput. Questi dispositivi sono collegati al
processore da un dispositivo di comunicazione chiamato bus. I dispositivi i I/O possono essere classificati in base alle operazioni che compiono: read o write, in base
al loro partner (uomo o macchina) e in base alla velocit\`a di trasferimento. 
\section{Connesione tra processori e periferiche}
Le connessioni avvengono tramite delle strutture di comunicazione chiamate bus che si possono dividere in due categorie:
\begin{enumerate}
\item Bus processore-memoria, specializzato, corto e veloce.
\item Bus I/O, possono essere lunghi e permettere la connessione con periferiche eterogenee, tipicamente non sono collegati direttamente alla memoria ma richiedono un
bus processore-memoria o un bus di sistema.
\end{enumerate}
Se prima era presente un unico bus parallelo che collegava tutto, ora per problemi di clock e frequenze troppo elevate si usano architetture composte da pi\`u bus 
paralleli condivisi e bus seriali punto-punto. Si indicher\`a con transizione I/O invio di indirizzo e spedizione o ricevimento di dati, con input il trasferimento di 
dati da una periferica verso una memoria dove il processore pu\`o leggerla, con output il trasferimento di dati dalla memoria al dispositivo. 
\subsection{Bus sincrono}
Tra le linee di controllo di questo bus esiste un clock e le comunicazioni avvengono con un protocollo collegato al il ciclo di clock. Questo tipo di bus \`e molto 
semplice da implementare e molto veloce, ma presenta poca robustezza al drift del clock e necessita che tutte le periferiche vadano alla velocit\`a del clock. 
\subsection{Bus asincrono}
Per rimediare agli inconvenienti del bus sincrono si utilizza il bus asincrono, in cui tutte le transazioni sono governate da una serie di handshake e non pi\`u dal 
clock. Questo richiede l'introduzione di nuove linee di controllo per segnalare inizio e fine di transazioni ma permette di collegare periferiche con velocit\`a diversa. 
Questo tipo di bus \`e robusto rispetto ai ritardi e consente di comunicare con periferiche di tipo diverso, ma \`e lento nelle connessioni in quanto richiede diversi 
segnali di controllo che devono circolare per permettere il passaggio di informazioni, la circuiteria di controllo \`e inoltre complessa. Si utilizzano spesso tecnologie
ibride, ma prevalentemente asincrone. Il trend pi\`u recente \`e quello di implementare all'interno del processore gli hub per il controllo di I/O e della memoria. 
\section{Prospettiva del programmatore}
Al programmatore interessa come trasformare una richiesta di I/O in un comando per la periferica e come trasferire i dati. Riguardo al sistema operativo occorre 
osservareche i programmi che condividono il processore condividono anche il sistema di I/O, i trasferimenti dati utilizzano delle interrupt che impattano le funzionalit
\`a del SO, devono pertanto essere eseguiti in una modalit\`a del processore detta supervisor cui solo il codice kernel pu\`o accedere, il controllo di I/O si interseca 
con problematiche di concorrenza. Un sistema operativo deve garantire che un utente con i permessi possa accedere ai sistemi di I/O cui pu\`o accedere, deve fornire 
comandi di alto livello per gestire operazioni di basso livello, gestire le interruzioni generate dall' I/O, ripartire l'accesso a un dispositivo tra i programmi che lo 
richiedono. Per implementare tali funzionalit\`a occorre rendere possibile al SO di inviare comandi alle periferiche, rendere possibile ai dispositivi notificare il 
corretto avvenimento di un'operazione, consentire trasferimenti diretti tra dispositivi e memoria. Questo si fa fornendo sulle relative linee di bus delle parole di 
controllo scrivendo o leggendo in particolari locazioni di memoria (memory mapped I/O) o tramite delle istruzioni speciali legate all'I/O. Scrivendo una particolare 
parola in una locazione di memoria associata al dispositivo il sistema di memoria ignora la scrittura e il controllore di I/O intercetta l'indirizzo particolare e 
trasmette il dato al dispositivo sotto forma di comando. Queste particolari locazioni di memoria sono inaccessibili ai programmi utante ma solo al sistema operativo,
pertanto deve esserci una chiamata di sistema che faccia commutare il processore in modalit\`a supervisore. Il dispositivo pu\`o utilizzare queste locazioni per 
trasmettere dati o pre-segnalare il suo stato. 
\section{Trasmettere o ricevere dati}
Per trasferire dati la maniera pi\`u semplice \`e l'attesa attiva o polling: si manda un comando di lettura scrittura alla periferica e si fa un ciclo di attesa testando
il bit di stato per vedere quando il dispositivo \`e pronto.
\subsubsection{Costo del polling}
Percentuale del processore utilizzata dal polling: prodotto tra frequenza di operazione e clock utilizzati dall'operazione di polling diviso la frequenza del processore.
\subsection{Considerazioni}
L'attesa attiva fa perdere tempo al processore che si dedica a cicli di lettura inutili, pu\`o pertanto essere utilizzato quando le operazioni avvengono con velocit\`a
determinata e se i dati vengono trasferiti con bitrate elevati il ciclo di attesa attiva dura poco. Se lo spreco \`e inaccettabile viene utilizzato I/O a interruzione
di programma (interrupt driven I/O). 
\subsection{Interruzioni di programma}
Un'interruzione I/O \`e un segnale utilizzato per segnalare al processore che la periferica \`e pronta a eseguire il trasferimento richiesto, possono avere diverso grado
di urgenza e occorre un modo per segnalare al processore quale periferica richiede l'interruzione. Questo tipo di interruzioni sono sempre asincrone rispetto 
all'esecuzione delle istruzioni. Quando arriva un'interruzione l'istruzione corrente viene terminata prima di considerare l'interruzione. Si pu\`o differire l'esecuzione
di un programma a un altro momento. Con questo metodo non occorre interrompere l'esecuzione del programma se non quando il dato pu\`o essere effettivamente riferito in
memoria, ma si necessita di hardware speciale per interrompere l'esecuzione del programma per permettere alle periferiche di generare un'interruzione e per rilevare 
l'istruzione, salvare lo stato del processore per esegure una routine di servizio (Interrupt Service Routine ISR) e poi riprendere dal punto dove si era interrotto. Il 
guadagno rispetto al polling \`e che l'overhead necessario alla lettura dei dati viene pagato solo quando il dato \`e effettivamete presente. 
\section{Eccezioni}
Un'eccezione \`e un trasferimento del controllo del programma non programmato. Il sistema effettua delle azioni per gestire le eccezioni, ome per esempio deve sapere
dove registrare il punto di interruzione e dove salvare lo stato e poi, ad eccezione finita, come riprendere dal punto immediatamente successivo al punto di 
interruzione. 
\subsection{Interrupts}
Sono eccezioni causate da eventi esterni, sono asincrone, possono essere gestite nello spazio tra due istruzioni, sospendono il programma e riprendono dallo stato in
cui si era interrotto. 
\subsection{Traps}
Sono eccezioni causate da eventi interni al programma, sincrone, gestite da un trap handler \`e possibile riprovare a eseguire l'istruzione che ha generato l'istruzione
o abortire il programma. 
