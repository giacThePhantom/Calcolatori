\chapter{Toolchain}
La toolchain \`e lo strumento necessario a tradurre il codice scritto in linguaggio di alto livello in linguaggio macchina: per esempio nel C ci sono quattro componenti
che permettono questa traduzione:
\begin{enumerate}
\item Preprocessore: gestisce le direttive \# (come include o define) sostituendo pezzi di codice ed eliminando i commenti. 
\item Compilatore: traduce il codice da C ad Assembly (.s).
\item Assembler: traduce da assembly a linguaggio macchina (.o).
\item Linker: collega tra loro diversi file .o per linkare al file oggetto librerie o altri file .o e produce un eseguibile. 
\end{enumerate}
L'eseguibile successivamente viene chiamato in memoria attraverso una system call per essere eseguito. 
\section{Da codice sorgente a eseguibile}
\subsection{Da codice sorgente a file oggetto}
Quando viene compilato il codice sorgente e trasformato in assembly questo viene ottimizzato e semplificato in modo da essere pi\`u performante e leggero. 
Successivamente il file assembly viene trasformato in un file oggetto. Durante questo passaggio le pseudo istruzioni vengono convertite in istruzioni standard, le 
istruzioni assembly in linguaggio macchina, tutti i numeri in binario le label tradotte in indirizzi, vengono gestiti i salti (se l'indirizzo non pu\`o essere contenuto 
dal campo j si passa ad un jr), vengono creati i metadati, le informazioni di alto livello che serviranno al loader per caricare il codice binario. Dopo aver completato 
queste operazioni il file oggetto contiene:
\begin{itemize}
\item Header: contiene dati come le posizioni degli altri dati all'interno del file.
\item Segmenti: contiene codice e dati come le variabili globali.
\item Tabella di rilocazione: contiene tutti i simboli con indirizzo relativo alla posizione nell'area .data e tipo di istruzione.
\item Tabella dei simboli: contiene tutti i simboli undefined non contenuti nel file oggetto e simboli definiti con indirizzo assoluto.
\item Altre informazioni come per esempio per il debugging. 
\end{itemize}
\subsection{Da file oggetto a eseguibile}
Durante questa traduzione viene deciso come codice e dati sono disposti in memoria compattando tra loro le sezioni con funzioni uguali, vengono associati indirizzi 
assoluti a tutti i simboli anche non locali, patcha le istruzioni di salto dopo che gli indirizzi sono stati modificati durante la prima fase della traduzione. Vengono
pertanto eliminate le tabelle dei simboli e di rilocazione sostituendo tutti gli indirizzi con indirizzi assoluti. Per fare ci\`o si potrebbe necessitare di collegare
tra loro pi\`u file .o. Durante questo processo si trovano tre tipi di simboli:
\begin{enumerate}
\item Locali definiti e visibili solamente all'interno del file.
\item Definiti associati ad un indirizzo relativo nella tabella dei simboli ma definiti all'interno del file stesso.
\item Non definiti presenti nella tabella dei simboli ma definiti in un file diverso. 
\end{enumerate}
\subsubsection{Linking}
Durante il linking vengono disposti in memoria i vari segmenti riordinandoli, ovvero prendendo tutte le parti .text nei file .o linkati e unificandole e con tutte le
altre sezioni, in modo da avere un unico file che contiene tutto il codice necessario e che sia strutturalmente ordinato. Successivamente si assegna un indirizzo 
assoluto ad ogni simbolo presente nella tabella dei simboli, facendo attenzione alla modifica degli indirizzi relativi che avviene nel primo passaggio. Alla fine si 
modificano tutte le istruzioni con gli indirizzi appena calcolati, sistemando tutti i simboli nella tabella di rilocazione. Il file risultante viene incapsulato in un 
eseguibile che conterr\`a i vari segmenti, informazioni per il caricamento in memoria e informazioni aggiuntive per il debugging. 
\section{Librerie}
Data la complessit\`a dei programmi si \`e reso necessario l'uso di librerie, collezioni di file .o per funzioni gi\`a create. Si dividono in:
\begin{itemize}
\item Librerie statiche (.a): collezioni di file .o. Il linker linka l'intera libreria. Dispendioso dal punto di vista della memoria, ma la fase di esecuzione risulta 
semplice.
\item Librerie dinamiche (.so): il linking avviene durante il caricamento ed esecuzione: la libreria pu\`o essere consultata a runtime. L'eseguibile rimane leggero, ma
presenta delle difficolt\`a di implementazione. 
\end{itemize}
\subsection{Librerie dinamiche}
Al momento del linking non viene linkata la libreria, ma solo dei riferimenti ad essa, e un riferimento ad un linker dinamico che viene caricato ed eseguito al momento
dell'esecuzione passandogli come argomento il programma. Sar\`a tale linker a linkare le librerie quando necessario. Oltre ad un eseguibile leggero permette di 
modularizzare il codice in modo che la ricompilazione non sia necessaria in caso di aggiornamento di una libreria. Il processo di caricamento diventa per\`o complesso.
Si pu\`o effettuare un lazy linking, in cui il linking viene effettuato solo quando strettamente necessario. In questa strategia invece dell'indirizzo della libreria
viene inserito uno stub che solo se invocato va a linkare la libreria. 