\chapter{Assembly intel}
Si basa su un'architettura CISC
\section{Gestione dei registri}
Per indicare un registro si prepone \%, per un valore immediato \$.Intel posside 16 registri a 64 bit general purpose: \%rax, \%rbx, \%rcx, \%rdx, \%rsi, \%rdi, \%r8, 
\%r9, \%r10, \%r11, \%r12, \%r13, \%r14, \%r15, \%rbp il frame pointer e \%rsp lo stack pointer. Per accedere ai 32 bit meno significativi si sostituisce r con e, per 
accedere ai 16 meno significativi si omette r, di questi 16 per accedere agli 8 pi\`u significatisi si prepone h, ai meno significativi l. Si trovano inoltre due registri 
specializzati: \%rip instrucion pointer (program counter) e \%rflags, ovvero il registro per i flags. 
\section{Convezioni di chiamata}
Il passaggio di parametri di una funzione \`e considerato da 6 registri: \%rsi, \%rdi, \%rcx, \%rdx, \%r8, \%r9, mentre ulteriori argomenti possono essere salvati sullo
stack. I valori di ritorno vengono salvati nei registri \%rax e \%rdx. I registri da preservare sono: \%rbp, \%rbx, \%r12, \%r13, \%r14, \%r15. Esistono due modalit\`a
per richiamare una funzione attraverso l'istruzione call: call address che chiama la subroutine all'indirizzo indicato e call *register che chiama la funzione 
all'indirizzo contenuto nel registro indicato. Il link register viene salvato automaticamente sullo stack, facilitando le funzioni ricorsive. L'istruzione ret preleva
dallo stack l'indirizzo di ritorno da utilizzare nell'instrucion pointer. 
\section{Modalit\`a di indirizzamento}
Esistono varie modalit\`a di indirizzamento in Intel:
\begin{itemize}
\item $<$displacement$>$
\item [$<$displacement$>$]($<$base register$>$)
\item [$<$displacement$>$]($<$index register$>$, [$<$scale$>$])
\item [$<$displacement$>$]($<$index register$>$, $<$base register$>$, [$<$scale$>$])
\end{itemize}
L'indirizzo di memoria corrispondente viene calcolato come $<$displacement$>+<$base$>+<$index$>\cdot<$scale$>$. Dove displacement \`e una costante a 8, 16, 32 o 64 bit, 
index e base sono indirizzi e scale \`e 1, 2, 4 o 8.
\section{Sintassi istruzioni}
Le istruzioni si presentano nella forma $<$opcode$>$[$<$size$>$]$<$source$>$, $<$destination$>$, dove il secondo elemento \`e sia operando che destinazione e deve essere
un registro o un indirizzo di memoria. Il primo pu\`o essere un valore immediato. I due operandi non possono essere contemporaneamente indirizzi in memoria e non \`e
possibile specificare due operandi e una destinazione diversa. $<$size$>$ viene utilizzato per determinare la grandezza degli operandi: b per 8 bit, w per 16, l per 32,  
q per 64. \`E opzionale quando uno dei due operandi \`e un registro, obbligatorio altrimenti. 
\section{Istruzioni frequenti}
A seguire un elenco delle istruzioni aritmetico-logiche più frequenti:
\begin{itemize}
\item mov per scambiare dati fra memoria e registri e viceversa (la destinazione non può essere un valore immediato);
\item push e pop per leggere e salvare dati sullo stack, senza dover modificare lo stack pointer come avviene in MIPS;
\item add (somma) e addc (somma con carry);
\item sub (sottrazione) e subc (sottrazione con carry);
\item mul (moltiplicazione con segno) e imul (moltiplicazione senza segno);
\item div (divisione con segno) e idiv (divisione senza segno);
\item inc (somma 1) e dec (sottrae 1);
\item and, or, xor e not: operazioni logiche bit a bit;
\item lea: load effective address;
\item rcl, rcr, rol, ror: varie forme di rotate;
\item sal, sar, shl, shr: shift aritmetico e logico;
\item jmp: salto incondizionato;
\item je (jump if equal), jnz (jump if not zero), jc (jump if carry), jnc (jump if not carry);
\item neg;
\item cmp: setta i flag facendo una sottrazione, ma senza salvarne il risultato;
\item call e ret: indirizzo di ritorno sullo stack;
\item nop;
\item eventuali istruzioni condizionali.
\end{itemize}
\subsubsection{Load effective address}
Nonoestante questa operazione sia stata implementata per calcolare indirizzi con indirizzamento diverso senza effettuare accessi, risulta utile per effettuare la somma
tra due registri e salvare il risultato in un terzo: lea (\%rax, \%rbx), \%rcx. Nonostante la sintassi sia simile a quella di un indirizzamento non lo si sta 
effettuando.
\subsubsection{Incremento e decremento}
Utilizzate invece di add in quanto permettono di risparimiare 16 bit sull'operazione.  