\chapter{La pipeline}
Si \`e notato come compiere un'operazione per ciclo di clock \`e inefficiente, pertanto si utilizza una pipeline. Si noti come le istruzioni del MIPS hanno 5 fasi 
di esecuzione: prelievo dell'istruzione dalla memoria, lettura dei registri e decodifica dell'istruzione, esecuzione di un'istruzione o calcolo di un indirizzo, accesso
a un operando nella memoria dati, scrittura del risultato in un registro, verr\`a utilizzata pertanto una pipeline a 5 stadi. In una pipeline quando un istruzione termina
l'utilizzo di un elmento funzionale questo viene subito occupato dall'istruzione successiva. 
\subsubsection{Confronto di prestazione}
Il confronto pu\`o essere fatto come tempo fra due istruzioni=$\frac{Tempo\ tra\ due\ istruzioni\ senza\ pipeline}{numero\ stadi\ della\ pipeline}$.
\section{Vantaggi del RISC}
\begin{enumerate}
\item Essendo tutte le istruzioni della stessa lunghezza il prelievo \`e pi\`u facile.
\item Essendo i codici degli operandi in posizione fissa ci si pu\`o accedere leggendo il register file in parallelo con la decodifica dell'istruzione.
\item Gli operandi residienti in memoria sono utilizzabili solo da sw e lw, pertanto si pu\`o utilizzare la ALU per calcolare gli indirizzi. .
\item L'uso di accessi allineati permette agli accessi in memoria di avvenire in un solo ciclo impegnando un solo stadio della pipeline. 
\end{enumerate}
\section{Hazard}
In condizioni normali la pipeline permette di eseguire un'operazione per ciclo di clock, ma ci sono dei casi in cui questo non \`e possibile per il verificarsi di 
condizioni critiche.
\subsection{Hazard strutturali}
L'architettura del calcolatore rende impossibile l'esecuzione di alcune istruzioni in pipeline, ad esempio, disponendo di un'unica memoria non \`e possibile caricare 
istruzioni e prelevare operandi nello stesso ciclo di clok.
\subsection{Hazard sui dati}
Si verifica quando la pipeline deve essere messa in stallo per ottenere informazioni dagli stadi precedenti, come somme che utilizzando gli stessi operandi. Questo 
tipo di hazard blocca la pipeline per tre cicli di clock.
\subsubsection{Possibili soluzioni}
\begin{itemize}
\item In certi casi si pu\`o eliminare il problema a livello di compilatore invertendo delle istruzioni.
\item \`E utile osservare come non \`e necessario salvare il risultato, attraverso un operazione di operand forwarding o propagazione, in cui il risultato viene reso 
disponibile bypassando l'operazione di storage.
\end{itemize}
\subsection{Hazard sul controllo}
Si verifica in presenza di salti condizionati, supponendo di avere un circuito sofisticato che permette di calcolare l'indirizzo di salto si necessita comunque di saltare
uno o due cicli. Se la pipeline \`e troppo lunga questo stallo diventa troppo costoso. Vengono pertanto implementati dei circuiti che prevedano dei salti e fanno delle 
assunzioni su di essi, mettendo le operazioni nella pipeline in base ad esse e poi, se necessario si corregge. Si pu\`o considerare che il salto non venga utilizzato, 
o che il comportamento rimanga costante per esempio. 